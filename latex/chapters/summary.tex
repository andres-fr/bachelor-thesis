The present thesis explores the question, whether an end-to-end approach could be successful in automatically learning and performing r\=aga classification, in comparison to existent domain-specific approaches.\\

For that, it starts by a description of the used dataset and the motivations behind its curation in chapter \ref{about-ds}, followed by a brief overview on carnatic music in chapter \ref{about-car}, necessary to achieve a better understanding of the related work and to provide some basis to help the interpretation of the experimental results.\\

Chapter \ref{context} reports the domain-specific approaches that conform the state of the art in r\=aga classification, as well as some recent and successful end-to-end approaches in the related field of Music Genre Recognition (MGR), which serve in this thesis as reference to apply to the r\=aga classification problem.\\

Chapter \ref{about-ml} follows providing a technical basis useful to understand the implementation of the concepts explained in chapter \ref{implementation}.\\

The final chapters \ref{experiments} and \label{conclusion} report the experiments done, as well as the line of thought behind them including motivations, interpretations and possible follow-ups.
