Among the variety of topics within music classification using machine learning, automatic classification of carnatic r\=agas is an interesting and suitable task for a bachelor's thesis, because the task is relatively well defined in its both musical and technical dimensions, and its intrinsic difficulty involves the study of a variety of recent approaches.\\

The state of the art for such task shows a clear dominance of the domain-specific solutions, with no successful end-to-end approach reported. This poses the question, whether such an end-to-end approach could be possible for the current tools and data avaliable.\\

In order to explore that question, the experiments conducted in the context of the present thesis applied several convolutional neural networks to perform end-to-end supervised learning on the CompMusic carnatic corpus, refraining from applying such domain-specific solutions.\\

The results of the experiments were that the different tested setups were unable to learn any relevant features for this task, possibly due to the inefficience of the chosen representation for the given amount of data.
